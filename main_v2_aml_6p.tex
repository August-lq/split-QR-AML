% AML期刊elsarticle模板规范
\documentclass[3p]{elsarticle}
% Use the option review to obtain double line spacing
% \documentclass[final,1p,times,twocolumn,authoryear]{elsarticle}
% 必要宏包
\usepackage{amssymb}
\usepackage{amsmath}
\usepackage{graphicx}
\usepackage{multirow}
\usepackage{paralist}
\usepackage{color}
\usepackage{float}
\usepackage{nicematrix}
\usepackage{tikz}
\usetikzlibrary{matrix,decorations.pathreplacing}
\usepackage{arydshln}%虚线
\usepackage{algorithm} % 算法
\usepackage{algmatlab}

% elsarticle依赖包
\usepackage{natbib} % 引文处理
\usepackage{geometry} % 页边距设置
% \usepackage{fleqn} % 左对齐公式
\usepackage{graphicx} % 插入图形


% 其他数学环境
\newtheorem{theorem}{Theorem}[section]
\newtheorem{definition}[theorem]{Definition}
\newtheorem{lemma}[theorem]{Lemma}
\newtheorem{corollary}[theorem]{Corollary}
\newtheorem{proposition}[theorem]{Proposition}
\newtheorem{example}[theorem]{Example}
\newtheorem{remark}[theorem]{Remark}
\newenvironment{proof}{\noindent {\bf Proof.\ }}{\vspace{2ex}}

\numberwithin{equation}{section}

\biboptions{numbers,sort&compress}

\begin{document}

%-------------------- FRONTMATTER --------------------
\begin{frontmatter}

% 标题
\title{An Algorithm for QR Decomposition of Split Quaternion Matrices}

% 作者与机构
\author[inst1]{Qianqian Liu}
\author[inst1]{Xin Liu\corref{cor1}}
\ead{xiliu@must.edu.mo}
\author[inst1]{Jianhai Lin\fnref{fn1}}
\author[inst2]{Yang Zhang}

\cortext[cor1]{Corresponding author}
\tnotetext[fund]{This research is supported by Macao Science and Technology Development Fund (No. 0013/2021/ITP), the grants from the National Natural Science Foundation of China (12371023, 12271338), and the Natural Sciences and Engineering Research Council of Canada (NSERC) (RGPIN 2020-06746), The joint research and Development fund of Wuyi University, Hong Kong and Macao (2019WGALH20), Macau University of Science and Technology Faculty Research Grants (FRG-22-073-FIE).}

\address[inst1]{Faculty of Innovation Engineering, Macau University of Science and Technology, Avenida Wai Long, TaiPa, Macau, 999078, P. R. China}
\address[inst2]{Department of Mathematics, University of Manitoba, Winnipeg, MB, R3T 2N2, Canada}
\fntext[fn1]{The author designed and implemented Algorithms \ref{alg:Permutation}, \ref{alg:Permutation Optimization}, and \ref{alg:QR}.}

% 摘要
\begin{abstract}
Split quaternions contain zero divisors, thus it is not a Euclidean distance space. Therefore, the traditional QR decomposition based on Givens rotations and Householder reflection transformations is difficult to implement. To overcome this difficulty and to address the non-commutativity of split quaternion multiplication, we will utilize the real representation $A^\sigma$ of the split quaternion matrix $A$.  By leveraging the proposed decomposition $A^\sigma = \widetilde{Q}R_4$  ($\widetilde{Q}$ is an orthogonal matrix, and $R_4 = \begin{bmatrix} R_{11} & R_{12} \\ R_{21} & R_{22} \end{bmatrix}$, with $R_{11}, R_{12}, R_{21}, R_{22}$ being upper triangular), the QR decomposition of the split quaternion matrix was successfully constructed using its real representations. To verify its effectiveness, the proposed algorithm was used for QR decomposition and solving a matrix equation, and the experimental results demonstrate its high efficiency in CPU time and accuracy.
\end{abstract}

% 关键词
\begin{keyword}
Split quaternion matrix \sep QR decomposition \sep Permutation matrix \sep Upper triangular matrix
\end{keyword}

\end{frontmatter}

%-------------------- 正文主体 --------------------

\section{Introduction}
In 1849, James Cockle \cite{Cockle1849} introduced the concept of split quaternion algebra over the real number field $\mathbb{R}$,   which is defined as
\begin{equation*}
    \small
    \mathbb{H}_s = \left\{ a_0 + a_1 i + a_2 j + a_3 k \ \bigg| i^2 = -1,\ j^2 = k^2 = 1, \ ijk = 1, \ a_0, a_1, a_2, a_3 \in \mathbb{R}\right\}. 
\end{equation*} 
$\mathbb{H}_s$ is a four dimensional non-commutative algebra and contains zero divisors. 
Split quaternion matrices have found applications in quantum mechanics, electromagnetism and signal processing (see, e.g.,\cite{Gog2022, Hasebe2010, Le2022, Z2022, Wang2023}). In the past decades, many studies have been done on the \iffalse algebraic properties of \fi split quaternions (see, e.g., \cite{Abłamowicz2020, Yasemin2012, TJiang2015, Jiang2018, TJiang2018, Zhuo2020, Yang2020, mma, wang, Wang2021, Gang2024, yuan, Zhang2015}). For instance, \cite{Jiang2018} studied the eigenvalue problem of split quaternion matrices; \iffalse \cite{Xin2019} derived a new real representation of split quaternion matrices to explain the consistency of two types of split quaternion matrix equations \(AX^* - XB = CY + D\) and \(X - AX^*B = CY + D\); \fi \cite{wang} solved the classical system of matrix equations; 
 \cite{Wang2021} proposed a fast algorithm for LDU decomposition; \cite{Gang2024} developed an efficient algorithm for the SVD decomposition. However, existing research has yet to fully address the theoretical development and algorithmic implementation of QR decomposition for split quaternion matrices.

To address the above problem, we constructively prove the existence of QR decomposition for split quaternion matrices and propose a novel, efficient algorithm for its computation.  To validate the high efficiency of our algorithm, we provide the experimental results in terms of speed and accuracy.

\section{Preliminaries}
For any split quaternion matrix ${A}=A_{0}+A_{1}i + A_{2}j + A_{3}k \in\mathbb{H}_{s}^{m\times n}$, where $A_{i}\in\mathbb{R}^{m\times n}, i\in\{0,1,2,3\}$, its transpose, conjugate, conjugate transpose, i-conjugate and i-conjugate transpose are  denoted by 
 ${A}^T = A_0^T + A_1^Ti + A_2^Tj + A_3^Tk,\bar{{A}} = A_0 - A_1i - A_2j - A_3k, {A}^* = A_0^T - A_1^Ti - A_2^Tj - A_3^Tk,$ and
 $\tilde{A} = A_0 - A_1i + A_2j + A_3k,{A}^H = A_0^T - A_1^Ti + A_2^Tj + A_3^Tk$, respectively. Recall that the real representation matrix of a split quaternion matrix $A \in\mathbb{H}_{s}^{m\times n}$ can be presented as:
\begin{equation}\label{eq:2.1}
A^\sigma = \begin{bmatrix} A_0 + A_2 & -A_1 + A_3 \\ A_1 + A_3 & A_0 - A_2 \end{bmatrix} \in \mathbb{R}^{2m \times 2n}.
\end{equation}
For $A, B \in \mathbb{H}_s^{m \times n}$, $C \in \mathbb{H}_s^{n \times p}$, $a \in \mathbb{R}$. It has the properties
\begin{equation}\label{eq:2.2}
    (A + B)^\sigma = A^\sigma + B^\sigma, \quad (AC)^\sigma = A^\sigma C^\sigma, \quad (a A)^\sigma = a A^\sigma, \ (A^H)^\sigma = (A^\sigma)^T.
\end{equation}
Conversely, for any real matrix $B = \begin{bmatrix} B_{11} & B_{12} \\ B_{21} & B_{22} \end{bmatrix} \in \mathbb{R}^{2m \times 2n}$, $B_{ts} \in \mathbb{R}^{m \times n}$, $t, s = 1, 2$, a corresponding split quaternion matrix $A$ can be constructed as follows:
\begin{equation}\label{eq:2.3}
{A} = \frac{B_{11} + B_{22}}{2} + \frac{B_{21} - B_{12}}{2}i + \frac{B_{11} - B_{22}}{2}j + \frac{B_{21} + B_{12}}{2}k.
\end{equation}
From equation \eqref{eq:2.1}, we have ${A}^\sigma = B$. 
That is,  $\sigma$ provides a one-to-one corresponding between $\mathbb{H}_s^{m\times n}$ and $\mathbb{R}^{2m \times 2n}$. We refer to \cite{Gang2024} for more details. Furthermore,  $A$ is called unitary if $AA^H = A^H A = I$. $A$ is unitary if and only if  $A^\sigma$ is  orthogonal (\cite{Gang2024}).
 The Frobenius norm of $A$ is defined as: 
 \begin{align*}
 % $
     \| A \|_F \equiv \frac{1}{\sqrt{2}} \| A^\sigma \|_F = \sqrt{\| A_0 \|_F^2 + \| A_1 \|_F^2 + \| A_2 \|_F^2 + \| A_3 \|_F^2}.
% $
\end{align*}
\iffalse  When $U, V$ are unitary, $U^\sigma$ and $V^\sigma$ are orthogonal, which will not change the Frobenius norm of $A^\sigma$. Hence
\begin{align*}
 % $
\|UAV\|_F = \frac{1}{\sqrt{2}} \|(UAV)^\sigma\|_F 
= \frac{1}{\sqrt{2}} \|U^\sigma A^\sigma V^\sigma\|_F 
= \frac{1}{\sqrt{2}} \|A^\sigma\|_F 
= \|A\|_F.
 % $
\end{align*}
\fi
\section{QR Decomposition of Split Quaternion matrix}
In this section, we discuss the QR decomposition for split quaternion matrices through construction, and develop an efficient algorithm. Let $A \in \mathbb{H}_s^{m \times n}$, we use the following two steps to construct its QR decomposition.  

\textbf{Step 1:} Perform the decomposition of $A^\sigma \in \mathbb{R}^{2m \times 2n}$ as 
\begin{equation}\label{splitqr}
A^\sigma = \widetilde{Q} R_4,
\end{equation} 
where $\widetilde{Q} \in \mathbb{R}^{2m\times 2m}$ is orthogonal, and $R_4  $ is in the form of 
\begin{equation}\label{r4}
R_4 = \begin{bmatrix}
    R_{11} & R_{12} \\
    R_{21} & R_{22}
\end{bmatrix} \in \mathbb{R}^{2m \times 2n},
\end{equation}
and $R_{11}, R_{12},R_{21},R_{22}$ are all upper triangular matrices of size $m \times n$. To achieve the special decomposition \eqref{splitqr},  we will first figure out how to transform an upper triangular matrix $R$ into $R_4$ through permutation transformations, as shown in the Figure \ref{fig:Upper triangular}.
\begin{figure}[htbp]
    % \begin{minipage}[htbp]{0.45\textwidth}
        \centering
        \includegraphics[width=0.45\textwidth,keepaspectratio=true]{Upper triangular.png} % Replace with actual file name
         % \caption{Transformation of a $2n \times 2n$ upper triangular matrix.}
    % \end{minipage}
    \caption{From $R$ to $R_4$ by permutations }
     \label{fig:Upper triangular}
\end{figure}
It is the key to accomplishing the special decomposition \eqref{splitqr}.
To better understand the procedure, we will choose the matrix 
\[R= \begin{bmatrix}
 r_{11} & r_{12} & r_{13} & r_{14} & r_{15} & r_{16} & r_{17} & r_{18}\\
 0      & r_{22} & r_{23} & r_{24} & r_{25} & r_{26} & r_{27} & r_{28}\\
 0      & 0      & r_{33} & r_{34} & r_{35} & r_{36} & r_{37} & r_{38}\\
 0      & 0      & 0      & r_{44} & r_{45} & r_{46} & r_{47} & r_{48}\\
 0      & 0      & 0      & 0      & r_{55} & r_{56} & r_{57} & r_{58}\\
 0      & 0      & 0      & 0      & 0      & r_{66} & r_{67} & r_{68}\\
\end{bmatrix}
\]
to demonstrate. 

We  perform the following row (r\#) and column (r\#) swaps in matrix $R$: 
\iffalse 
  Replace the odd rows (1, 3, 5) sequentially with the first three rows. Replace the even rows (2, 4, 6) sequentially with the last three rows. Replace the odd columns (1, 3, 5) sequentially with the first three columns. Replace the even columns (2, 4, 6) sequentially with the last three columns.\fi
{\color{red}Place the rows (1, 3, 5, 2, 4, 6) of the $R$ matrix respectively in the (1, 2, 3, 4, 5, 6) rows of the resulting matrix.  Next, perform column operations on the resulting matrix, placing the columns (1, 3, 5, 7, 2, 4, 6, 8) of the resulting matrix respectively in the columns (1, 2, 3, 4, 5, 6, 7, 8). In this way, we can obtain}
\[R_4 = \begin{bmatrix}
\begin{array}{cccc:cccc}
 r_{11} & r_{13} & r_{15} & r_{17} & r_{12} & r_{14} & r_{16} & r_{18}\\
 0      & r_{33} & r_{35} & r_{37} & 0      & r_{34} & r_{36} & r_{38}\\
 0      & 0      & r_{55} & r_{57} & 0      & 0      & r_{56} & r_{58}\\
 \cdashline{1-8}
 0      & r_{23} & r_{25} & r_{27} & r_{22} & r_{24} & r_{26} & r_{28}\\
 0      & 0      & r_{45} & r_{47} & 0      & r_{44} & r_{46} & r_{48}\\
 0      & 0      & 0      & r_{67} & 0      & 0      & r_{66} & r_{68}\\
\end{array}
\end{bmatrix}=\begin{bmatrix}
    R_{11} & R_{12}\\R_{21} & R_{22}
\end{bmatrix}.
\]
\iffalse
\begin{align*}
R
& \xrightarrow{r_{2} \leftrightarrow r_{5}}
\begin{bmatrix}
% \begin{array}{cc:cc:cc}
 r_{11} & r_{12} & r_{13} & r_{14} & r_{15} & r_{16}\\
 0      & 0      & 0      & 0      & r_{55} & r_{56}\\
 0      & 0      & r_{33} & r_{34} & r_{35} & r_{36}\\
 \cdashline{1-6}
 0      & 0      & 0 & r_{44} & r_{45} & r_{46}\\
 0 & r_{22} & r_{23} & r_{24} & r_{25} & r_{26}\\
 0      & 0      & 0      & 0      & 0 & r_{66}\\
% \end{array}
\end{bmatrix}
\xrightarrow{r_{2} \leftrightarrow r_{3}}
\begin{bmatrix}
% \begin{array}{cc:cc:cc}
 r_{11} & r_{12} & r_{13} & r_{14} & r_{15} & r_{16}\\
 0      & 0      & r_{33} & r_{34} & r_{35} & r_{36}\\
 0      & 0      & 0      & 0      & r_{55} & r_{56}\\
 \cdashline{1-6}
 0      & 0      & 0 & r_{44} & r_{45} & r_{46}\\
 0 & r_{22} & r_{23} & r_{24} & r_{25} & r_{26}\\
 0      & 0      & 0      & 0      & 0 & r_{66}\\
% \end{array}
\end{bmatrix} \\
& \xrightarrow{r_{4} \leftrightarrow r_{5}}
\begin{bmatrix}
% \begin{array}{cc:cc:cc}
 r_{11} & r_{12} & r_{13} & r_{14} & r_{15} & r_{16}\\
 0      & 0      & r_{33} & r_{34} & r_{35} & r_{36}\\
 0      & 0      & 0      & 0      & r_{55} & r_{56}\\
 \cdashline{1-6}
 0 & r_{22} & r_{23} & r_{24} & r_{25} & r_{26}\\
 0      & 0      & 0 & r_{44} & r_{45} & r_{46}\\
 0      & 0      & 0      & 0      & 0 & r_{66}\\
% \end{array}
\end{bmatrix}
\xrightarrow{c_{2} \leftrightarrow c_{5}}
\begin{bmatrix}
\begin{array}{ccc:ccc}
 r_{11} & r_{15} & r_{13} & r_{14}  & r_{12} & r_{16}\\
 0      & r_{35} & r_{33} & r_{34}  & 0      & r_{36}\\
 0      & r_{55} & 0      & 0       & 0      & r_{56}\\
 \cdashline{1-6}
 0      & r_{25} & r_{23} & r_{24}  & r_{22} & r_{26}\\
 0      & r_{45} & 0      & r_{44}  & 0      & r_{46}\\
 0      & 0      & 0      & 0       & 0      & r_{66}\\
\end{array}
\end{bmatrix} \\
& \xrightarrow{c_{2} \leftrightarrow c_{3}}
\begin{bmatrix}
\begin{array}{ccc:ccc}
 r_{11} & r_{13} & r_{15} & r_{14}  & r_{12} & r_{16}\\
 0      & r_{33} & r_{35} & r_{34}  & 0      & r_{36}\\
 0      & 0      & r_{55} & 0       & 0      & r_{56}\\
 \cdashline{1-6}
 0      & r_{23} & r_{25} & r_{24}  & r_{22} & r_{26}\\
 0      & 0      & r_{45} & r_{44}  & 0      & r_{46}\\
 0      & 0      & 0      & 0       & 0      & r_{66}\\
\end{array}
\end{bmatrix}
\xrightarrow{c_{4} \leftrightarrow c_{5}}
\begin{bmatrix}
\begin{array}{ccc:ccc}
 r_{11} & r_{13} & r_{15} & r_{12} & r_{14} & r_{16}\\
 0      & r_{33} & r_{35} & 0      & r_{34} & r_{36}\\
 0      & 0      & r_{55} & 0      & 0      & r_{56}\\
 \cdashline{1-6}
0 & r_{23} & r_{25} & r_{22} & r_{24} & r_{26}\\
 0      & 0 & r_{45} & 0      & r_{44} & r_{46}\\
 0      & 0      & 0 & 0      & 0      & r_{66}\\
\end{array}
\end{bmatrix} \\
&=\begin{bmatrix}
    R_{11} & R_{12}\\R_{21} & R_{22}
\end{bmatrix} = R_4
\end{align*}
\fi
\iffalse 
{\color{red}Perform an ordered swap between the even-indexed rows in the upper half and the odd-indexed rows in the lower half of the matrix, then partition it into two upper and lower submatrices (for submatrices with an odd number of rows, appending zeros to the end of the last row of the upper submatrix and prepending zeros to the beginning of the first row of the lower submatrix ensures the parity of each row index remains unchanged before the swap, the position of zero rows stays fixed after the swap, and further partitioning is possible). Subsequently, the above-mentioned row swapping and partitioning operations are recursively applied to each submatrix until each submatrix contains only two rows; analogous operations are performed on the columns, ultimately producing four upper triangular block matrices.}\\
We perform a global row swap \( r_{2} \leftrightarrow r_{5} \), $Step1$, partition the matrix $R$ into upper and lower blocks and swap rows \( r_{2} \leftrightarrow r_{3} \), $Step2.1$, in the upper block and \( r_{4} \leftrightarrow r_{5} \), $Step2.2$, in the lower block, then perform analogue column operations, $Step3-4$. Demonstrate below:
% We can obtain the following resulting matrix:
\begin{align*}
R
& \xrightarrow[Step1]{r_{2} \leftrightarrow r_{5}}
\begin{bmatrix}
% \begin{array}{cc:cc:cc}
 r_{11} & r_{12} & r_{13} & r_{14} & r_{15} & r_{16}\\
 0      & 0      & 0      & 0      & r_{55} & r_{56}\\
 0      & 0      & r_{33} & r_{34} & r_{35} & r_{36}\\
 \cdashline{1-6}
 0      & 0      & 0 & r_{44} & r_{45} & r_{46}\\
 0 & r_{22} & r_{23} & r_{24} & r_{25} & r_{26}\\
 0      & 0      & 0      & 0      & 0 & r_{66}\\
% \end{array}
\end{bmatrix}
\xrightarrow[Step2.1]{r_{2} \leftrightarrow r_{3}}
\begin{bmatrix}
% \begin{array}{cc:cc:cc}
 r_{11} & r_{12} & r_{13} & r_{14} & r_{15} & r_{16}\\
 0      & 0      & r_{33} & r_{34} & r_{35} & r_{36}\\
 0      & 0      & 0      & 0      & r_{55} & r_{56}\\
 \cdashline{1-6}
 0      & 0      & 0 & r_{44} & r_{45} & r_{46}\\
 0 & r_{22} & r_{23} & r_{24} & r_{25} & r_{26}\\
 0      & 0      & 0      & 0      & 0 & r_{66}\\
% \end{array}
\end{bmatrix} \\
& \xrightarrow[Step2.2]{r_{4} \leftrightarrow r_{5}}
\begin{bmatrix}
% \begin{array}{cc:cc:cc}
 r_{11} & r_{12} & r_{13} & r_{14} & r_{15} & r_{16}\\
 0      & 0      & r_{33} & r_{34} & r_{35} & r_{36}\\
 0      & 0      & 0      & 0      & r_{55} & r_{56}\\
 % \cdashline{1-6}
 0 & r_{22} & r_{23} & r_{24} & r_{25} & r_{26}\\
 0      & 0      & 0 & r_{44} & r_{45} & r_{46}\\
 0      & 0      & 0      & 0      & 0 & r_{66}\\
% \end{array}
\end{bmatrix}
\xrightarrow[Step3]{c_{2} \leftrightarrow c_{5}}
\begin{bmatrix}
\begin{array}{ccc:ccc}
 r_{11} & r_{15} & r_{13} & r_{14}  & r_{12} & r_{16}\\
 0      & r_{35} & r_{33} & r_{34}  & 0      & r_{36}\\
 0      & r_{55} & 0      & 0       & 0      & r_{56}\\
 0      & r_{25} & r_{23} & r_{24}  & r_{22} & r_{26}\\
 0      & r_{45} & 0      & r_{44}  & 0      & r_{46}\\
 0      & 0      & 0      & 0       & 0      & r_{66}\\
\end{array}
\end{bmatrix} \\
& \xrightarrow[Step4.1]{c_{2} \leftrightarrow c_{3}}
\begin{bmatrix}
\begin{array}{ccc:ccc}
 r_{11} & r_{13} & r_{15} & r_{14}  & r_{12} & r_{16}\\
 0      & r_{33} & r_{35} & r_{34}  & 0      & r_{36}\\
 0      & 0      & r_{55} & 0       & 0      & r_{56}\\
 % \cdashline{1-6}
 0      & r_{23} & r_{25} & r_{24}  & r_{22} & r_{26}\\
 0      & 0      & r_{45} & r_{44}  & 0      & r_{46}\\
 0      & 0      & 0      & 0       & 0      & r_{66}\\
\end{array}
\end{bmatrix}
\xrightarrow[Step4.2]{c_{4} \leftrightarrow c_{5}}
\begin{bmatrix}
\begin{array}{ccc:ccc}
 r_{11} & r_{13} & r_{15} & r_{12} & r_{14} & r_{16}\\
 0      & r_{33} & r_{35} & 0      & r_{34} & r_{36}\\
 0      & 0      & r_{55} & 0      & 0      & r_{56}\\
 \cdashline{1-6}
0 & r_{23} & r_{25} & r_{22} & r_{24} & r_{26}\\
 0      & 0 & r_{45} & 0      & r_{44} & r_{46}\\
 0      & 0      & 0 & 0      & 0      & r_{66}\\
\end{array}
\end{bmatrix} 
% &=\begin{bmatrix}
%     R_{11} & R_{12}\\R_{21} & R_{22}
% \end{bmatrix}
= R_4
\end{align*}
 \fi

\newpage
\begin{algorithm}[H]
    \caption{Matrix Permutation Recursive Algorithm} \label{alg:Permutation}
    \begin{algorithmic}[1]
    \Function{Permutation}{$A,\text{flag}$}
        \State \% \textbf{Input:} $A \in \mathbb{R}^{s \times 2n}$, with flag=0 for first submatrix, flag=1 for second.
        \State \% \textbf{Output:} $B \in \mathbb{R}^{s \times 2n}$.
        % \State \% \textbf{Step 1} Recursive stop condition:
        \If{$s=2$} \qquad\qquad\qquad\qquad\qquad\qquad\qquad\qquad\qquad\qquad\qquad \% \textbf{Step 1} Recursive stop condition
            \State $B=A$; \Return \ $B$;
        \End 
        % \State \% \textbf{Step 2} Initialization:
        \State $s2 = \text{floor}((s+1)/2)$; $s3 = \text{floor}(s2/2)$;\qquad\qquad\qquad\qquad\quad\% \textbf{Step 2} Initialization
        % \State $s3 = \text{floor}(s2/2)$;
        % \State \% \textbf{Step 3} Add zero rows:
        \If{$\text{mod}(s, 2)=1$} \qquad\qquad\qquad\qquad\qquad\qquad\qquad\qquad\qquad\ \ \% \textbf{Step 3} Add zero rows
            \If{flag=1}
                \State $A = [\text{zeros}(1, \text{size}(A, 2)); A]$;
            \Else
                \State $A = [A; \text{zeros}(1, \text{size}(A, 2))]$;
            \End
        \End
        % \State \% \textbf{Step 4} Perform swapping:
        \For{$i=1:s3$} \qquad\qquad\qquad\qquad\qquad\qquad\qquad\qquad\qquad\qquad\ \% \textbf{Step 4} Perform swapping
            \If{$\text{mod}(s2, 2)=0$}
                \State $A= \text{swap}(A,2*i,s2+2*(i-1)+1)$;
            \Else
                \State $A= \text{swap}(A,2*i,s2+2*(i-1)+2)$;
            \End
        \End
        % \State \% \textbf{Step 5} Recursive execution:
        \State $A(1:s2, :) = \text{Permutation}(A(1:s2, :), 0)$; \qquad\qquad\qquad\qquad\% \textbf{Step 5} Recursive execution
        \State $A(s2+1:\text{end}, :) = \text{Permutation}(A(s2+1:\text{end}, :), 1)$;
        % \State \% \textbf{Step 6} Remove zero rows:
        \If{$\text{mod}(s, 2)=1$} \qquad\qquad\qquad\qquad\qquad\qquad\qquad\qquad\qquad\quad \% \textbf{Step 6} Remove zero rows
            \If{flag=1}
                \State $B = A(2:\text{end}, :)$;
            \Else
                \State $B = A(1:\text{end}-1, :)$;
            \End
        \Else
            \State $B=A$;
        \End
    \End 
    \end{algorithmic}
\end{algorithm}

The above process applies to any upper triangular matrices $R$ with even orders $2m \times 2n$. By the properties of the row and column transformations, it is easy to check that the relationship between $R$ and $R_4$ is as follows:
\begin{equation}\label{eq:Rn}
    P_{m} R P_{n}^T = \begin{bmatrix} R_{11} & R_{12}\\R_{21}& R_{22}\end{bmatrix}.
\end{equation}

To obtain the matrix $P_k, k\in \{m,n\}$ for $R \in \mathbb{R}^{2m \times 2n}$, we established Algorithm \ref{alg:Permutation}. When the input is $A=I_{2k}$ and $flag=0$, we can obtain the permutation matrix $P_k$:

\begin{equation}\label{p}
    P_k = \begin{bmatrix} 
            1 & 0 & 0 & 0 & \cdots & 0 & 0\\ 
            0 & 0 & 1 & 0 & \cdots & 0 & 0\\ 
            \vdots & \vdots & \vdots & \vdots &  & \vdots & \vdots\\ 
            0 & 0 & 0 & 0 & \cdots & 1 & 0 \\
            0 & 1 & 0 & 0 & \cdots & 0 & 0\\ 
            0 & 0 & 0 & 1 & \cdots & 0 & 0\\ 
            \vdots & \vdots & \vdots & \vdots &  & \vdots & \vdots\\ 
            0 & 0 & 0 & 0 &\cdots & 0 & 1 
        \end{bmatrix}_{2k \times 2k}
\end{equation}


After completing the above steps, we can now proceed to implement the decomposition \eqref{splitqr}.
First, perform the QR decomposition:
$P_m^TA^\sigma P_n = \widehat{Q}\widehat{R}.$ 
According to \eqref{eq:Rn}, we have
$
P_m^TA^\sigma P_n = \widehat{Q}\widehat{R} = \widehat{Q}(P_m^TR_4P_n).
$
Multiplying both sides of the equation by $P_m$ on the left and $P_n^T$ on the right yields
$
A^\sigma=P_m\widehat{Q}P_m^T R_4.
$ Since $P_m$ is a permutation matrix, it is also an orthogonal matrix. Thus, the product of matrices, $P_m\widehat{Q}P_m^T$, is also an orthogonal matrix. Then we get our required special decomposition $A^\sigma=\widetilde{Q}R_4,$ where $\widetilde{Q}=P_m\widehat{Q}P_m^T$ is orthogonal, and $R_4$ is in the form of \eqref{r4}.
\begin{algorithm}[htbp]
    \caption{Matrix Permutation Optimization Algorithm}
    \label{alg:Permutation Optimization}
    \begin{algorithmic}[1]
    \Function{PermutationOpt}{$A,t$}
        \State \% \textbf{Input:} $ A\in \mathbb{R}^{2m\times 2n}$, $t$ is 0 or 1.
        \State \% \textbf{Output:} $B \in\mathbb{R}^{2m\times 2n}$, $B=PA$ when t is 0, and $B=P^TA$ when t is 1.
        \If {$t == 1$}
          \For {$i = 1:m$}
            \State $B(2*i-1, :) = A(i, :);$
            \State $B(2*i, :) = A(m+i, :);$
          \End
        \Else
          \For {$i = 1:m$}
            \State $B(i, :) = A(2*i-1, :);$
            \State $B(m+i, :) = A(2*i, :);$
          \End
        \End
    \End 
    \end{algorithmic}
\end{algorithm}
To reduce the computational cost, we designed Algorithm \ref{alg:Permutation Optimization}, which avoids multiplication operations on $\widehat{A}, \widehat{R}, \widehat{Q}$.

\textbf{Step 2:} Using equation \eqref{eq:2.3}, construct the matrix $R$ such that $R^\sigma=R_4:$
\begin{equation*}
R = \frac{R_{11} + R_{22}}{2} + \frac{R_{21} - R_{12}}{2}i + \frac{R_{11} - R_{22}}{2}j + \frac{R_{21} + R_{12}}{2}k,
\end{equation*}
Since $R_{11}, R_{12}, R_{21}, R_{22}$ are upper triangular, this ensures that the constructed split quaternion matrix $R$ is also upper triangular. Additionally, construct the matrix $Q$ such that $Q^\sigma=\widetilde{Q}:$
\begin{equation*}
Q = \frac{Q_{11} + Q_{22}}{2} + \frac{Q_{21} - Q_{12}}{2}i + \frac{Q_{11} - Q_{22}}{2}j + \frac{Q_{21} + Q_{12}}{2}k,
\end{equation*}
where $Q_{11}, Q_{12}, Q_{21}, Q_{22} \in \mathbb{R}^{m \times n}$ are the blocks of the orthogonal matrix $\widetilde{Q} = \begin{bmatrix} Q_{11} & Q_{12} \\ Q_{21} & Q_{22} \end{bmatrix} \in \mathbb{R}^{2m \times 2m}$. Note that $(Q^H Q)^\sigma = {(Q^H)}^\sigma Q^\sigma = {(Q^\sigma)}^TQ^\sigma = \widetilde{Q}^T\widetilde{Q} = I_{2m}$, indicating that the split quaternion matrix $Q$ remains a unitary matrix.
Now, $A^\sigma=\widetilde{Q}R_4=Q^\sigma R^\sigma$ implies
$A = Q R$, as required.
 
\begin{algorithm}[htbp] 
    \caption{Compute the QR of Split Quaternion Matrix \(A\)}
    \label{alg:QR}
    \begin{algorithmic}[1]
    \Function{Split-QR}{$A$}
        \State \% \textbf{Input:} \(A = A_0 + A_1 i + A_2 j + A_3 k \in \mathbb{H}_s^{m\times n}\).
        \State \% \textbf{Output:} Unitary matrix \(Q \in \mathbb{H}_s^{m\times m}\), upper triangular matrix \(R \in \mathbb{H}_s^{m\times n}\), satisfying \(Q  R = A\).
        % \State \% \textbf{Step 1} Real representation of $A$:
        \State \(A^\sigma = \begin{bmatrix}
            A_0 + A_2 & -A_1 + A_3 \\ 
            A_1 + A_3 & A_0 - A_2
            \end{bmatrix} \in \mathbb{R}^{2m\times 2n}\); \qquad\qquad\qquad\qquad \% \textbf{Step 1} Real representation of $A$
        
        % \State \% \textbf{Step 2} Compute \(\widehat{A} = P_{m}^T A^\sigma P_{n}\): 
        \State $\widehat{A}=\text{PermutationOpt}(A^\sigma,1);$  \qquad\qquad\qquad\qquad\qquad\qquad\quad\% \textbf{Step 2} Compute \(\widehat{A} = P_{m}^T A^\sigma P_{n}\)
        \State $\widehat{A}=\text{PermutationOpt}(\widehat{A}',1),\widehat{A}=\widehat{A}';$
        
        % \State \% \textbf{Step 3} Compute QR:
        \State \([\widehat{Q},\widehat{R}] = \text{qr}(\widehat{A})\); \qquad\qquad\qquad\qquad\qquad\qquad\qquad\qquad\qquad\quad\% \textbf{Step 3} Compute QR
        
        % \State \% \textbf{Step 4} Compute \(R^\sigma = P_{m}\widehat{R}P_{n}^T\), \(Q^\sigma =      P_{m}\widehat{Q}P_{m}^T\): 
        \State $\widehat{R}=\text{PermutationOpt}(\widehat{R},0)$; \qquad\qquad\qquad\qquad\qquad\qquad\quad\ \ \% \textbf{Step 4} Compute \(R^\sigma = P_{m}\widehat{R}P_{n}^T\), 
        \State $\widehat{R}=\text{PermutationOpt}(\widehat{R}',0);R^\sigma=\widehat{R}'$;\qquad\qquad\qquad\qquad\quad\ \ \% \qquad\quad \ \(Q^\sigma =      P_{m}\widehat{Q}P_{m}^T\)
        \State $\widehat{Q}=\text{PermutationOpt}(\widehat{Q},0)$;
        \State $\widehat{Q}=\text{PermutationOpt}(\widehat{Q}',0);Q^\sigma=\widehat{Q}'$;
        
        % \State \% \textbf{Step 5} Construct matrix Q, R:
        \State $Q_0 = (Q^\sigma(1\!:\!m,1\!:\!m) + Q^\sigma(m+1\!:\!2m,m+1\!:\!2m))/2$; \qquad\ \% \textbf{Step 5} Construct matrix Q, R
        \State $Q_1 = (Q^\sigma(m+1\!:\!2m,1\!:\!m) - Q^\sigma(1\!:\!m,m+1\!:\!2m))/2$;
        \State $Q_2 = (Q^\sigma(1\!:\!m,1\!:\!m) - Q^\sigma(m+1\!:\!2m,m+1\!:\!2m))/2$;
        \State $Q_3 = (Q^\sigma(m+1\!:\!2m,1\!:\!m) + Q^\sigma(1\!:\!m,m+1\!:\!2m))/2$;
        \State $R_0 = (R^\sigma(1\!:\!m,1\!:\!n) + R^\sigma(m+1\!:\!2m,n+1\!:\!2n))/2$;
        \State $R_1 = (R^\sigma(m+1\!:\!2m,1\!:\!n) - R^\sigma(1\!:\!m,n+1\!:\!2n))/2$;
        \State $R_2 = (R^\sigma(1\!:\!m,1\!:\!n) - R^\sigma(m+1\!:\!2m,n+1\!:\!2n))/2$;
        \State $R_3 = (R^\sigma(m+1\!:\!2m,1\!:\!n) + R^\sigma(1\!:\!m,n+1\!:\!2n))/2$;

        % \State \% \textbf{Step 6} Return: 
        \State $Q = Q_0 + Q_1i + Q_2j + Q_3k$; \qquad\qquad\qquad\qquad\qquad\qquad\qquad\% \textbf{Step 6} Return
        \State $R = R_0 + R_1i + R_2j + R_3k$;
    \End 
    \end{algorithmic}
\end{algorithm}

From the process described above,  any split quaternion matrix can be decomposed into QR decomposition. Our results can be summarized as follows:
\begin{theorem}(QR Decomposition)
    Let $A \in \mathbb{H}_s^{m \times n}$. There exist a unitary matrix $Q \in \mathbb{H}_s^{m \times m}$ and an upper triangular matrix $R \in \mathbb{H}_s^{m \times n}$ such that
    \begin{eqnarray}\label{eq:split QR}
        A = Q R.
    \end{eqnarray}
\end{theorem}
\iffalse
\textbf{Complexity Analysis:}
{Complexity Analysis:} For a split quaternion matrix $A = A_0 + A_1i + A_2j + A_3k \in \mathbb{H}_s^{m \times n}$, the computational complexity of Algorithm \ref{alg:QR} is dominated by three components:  

\textbf{1. Real Representation Conversion:}
Constructing the real representation $A^\sigma \in \mathbb{R}^{2m \times 2n}$ requires about $4mn$ flops. 

\textbf{2. Permutations On $\widehat{A}, \widehat{R}, \widehat{Q}$:}
By using Algorithm \ref{alg:Permutation Optimization},  which requires approximately  $(8m+4n)$ flops.

\textbf{3. Real Matrix QR Decomposition:}
The QR decomposition of $\widehat{A} \in \mathbb{R}^{2m \times 2n}$ using Householder reflections dominates the complexity at $16(mn^2-\frac{n^3}{3})$ flops.


\textbf{Total Complexity}
$$
\underbrace{(4mn)}_{\text{Real Rep.}} + \underbrace{(8m+4n)}_{\substack{\text{Permutation} \\ \text{(optimized)}}} + \underbrace{(16(mn^2-\frac{n^3}{3}))}_{\text{QR}} \approx \boxed{(16(mn^2-\frac{n^3}{3}))}
$$  
\fi

\section{Numerical Examples}
\iffalse In this section,  we will use the proposed  Algorithm 2 to compute the QR decomposition for split quaternion matrices and apply it to solving a matrix equation.\fi
In this section,  we will use the proposed  Algorithm 3 to compute the QR decomposition for split quaternion matrices.
\begin{example}
    Given a split quaternion matrix $A = A_{0}+A_{1}i+A_{2}j+A_{3}k\in \mathbb{H}_s^{m\times n}$, where
    \begin{equation}
       \begin{cases}
            m = 25,50,\cdots,500;
            n = 25,50,\cdots,500;  \\
            A_{0}=\text{rand}(m,n);
            A_{1}=\text{rand}(m,n); 
            A_{2}=\text{rand}(m,n);
            A_{3}=\text{rand}(m,n).
        \end{cases} \label{eq:example2}
    \end{equation}
\end{example}
Here, the MATLAB function $rand(m,n)$ is used to generate a $m \times n$ real matrix with random elements that fall within the interval [0,1]. \iffalse All experiments were performed and computed using MATLAB 2024a on a system equipped with an 11th Gen Intel(R) Core(TM) i7-1185G7 @ 3.00GHz processor, 16.0 GB RAM and Windows 11 Pro.\fi We executed Algorithm \ref{alg:QR} to perform the QR decomposition on $A$, and have calculated the CPU time and the relative error
$\epsilon = \frac{\left\|A - Q R\right\|_{F}}{\|A\|_{F}}.$
The experimental results from Figure \ref{fig:cpu times} show that Algorithm \ref{alg:QR} performs quite well in both speed and accuracy. 
\begin{figure}[htbp]
    \centering
    \begin{minipage}[b]{0.45\textwidth}
        \centering
        \includegraphics[width=\textwidth]{cpu times.png} % Replace with actual filename
       % \subcaption{(a)}
    \end{minipage}
    \hfill % Add space
    \begin{minipage}[b]{0.45\textwidth}
        \centering
        \includegraphics[width=\textwidth]{error.png} % Replace with actual filename
        % \subcaption{(b)}
    \end{minipage}
    % \captionsetup{font=footnotesize}
    \caption{ CPU Time and Error Analysis of the QR Algorithm for Split Quaternion Matrices }
     \label{fig:cpu times}
\end{figure}
\iffalse
In the following, we will discuss the application of QR decomposition in solving matrix equations.
\begin{example}
Utilize the QR algorithm to solve the split quaternion matrix equation $AX = B$, where
\begin{align*}
  & A =
    \begin{bmatrix}
    -4 & -2 & -8 \\
    -2 & -2 & -5 \\
     7 & -3 & -9
    \end{bmatrix} +
    \begin{bmatrix}
    -1 & -2 &  4 \\
    -5 & -8 & -5 \\
    -4 &  0 &  6
    \end{bmatrix} i 
    + 
    \begin{bmatrix}
    -9  & -6  & -8 \\
    -2  & -10 & -5 \\
    -10 & -7  & -7
    \end{bmatrix} j +
    \begin{bmatrix}
    -8 &  9 & -3 \\
     2 & -6 &  0 \\
     8 &  0 & -5
    \end{bmatrix} k,\\
% \end{align*}
% \begin{align*}
  & B =
    \begin{bmatrix}
    -9 & -10 &  10 \\
    -1 &  10 &  6 \\
    -7 & -1  & -10
    \end{bmatrix} +
    \begin{bmatrix}
    4 &  1 & -6 \\
    4 & -6 & -3 \\
    3 &  6 &  8
    \end{bmatrix} i 
    +
    \begin{bmatrix}
     7 & 2 &  2 \\
    -2 & 9 & -4 \\
    -4 & 9 &  7
    \end{bmatrix} j +
    \begin{bmatrix}
    -1 &   1 & -3 \\
     8 &   5 & -7 \\
    -10 & -7 & -4
    \end{bmatrix} k.
\end{align*}
\end{example}  

First, using Algorithm \ref{alg:QR} to perform QR decomposition on $A$ to obtain $Q$ and $R:$
\setlength{\jot}{2pt}
\setlength{\arraycolsep}{1pt}
{\footnotesize
\begin{align*}
  Q =
    & \begin{bmatrix}
    -0.514 & -0.047 & -0.344 \\
    -0.126 & -0.411 & -0.041 \\
    -0.436 & -0.429 & -0.099
    \end{bmatrix} +
    \begin{bmatrix}
    -0.363 &  0.479 &  0.122 \\
     0.077 & -0.458 & -0.100 \\
     0.383 & -0.194 &  0.138
    \end{bmatrix} i
    + 
    \begin{bmatrix}
    -0.236 & -0.062 & -0.155 \\
    -0.105 &  0.031 &  0.403 \\
     0.262 &  0.178 & -0.296
    \end{bmatrix} j +
    \begin{bmatrix}
    -0.157 &  0.171 &  0.319 \\
    -0.250 & -0.133 &  0.577 \\
    -0.152 &  0.291 & -0.343
    \end{bmatrix} k,\\
  R =
    & \begin{bmatrix}
    -3.041 & 3.419 & 10.927 \\
     0     & 6.061 & 6.456 \\
     0     & 0     & 7.714
    \end{bmatrix} +
    \begin{bmatrix}
    -1.443 & 8.617 &  1.053 \\
     0     & 1.847 & -2.947 \\
     0     & 0     & -5.025
    \end{bmatrix} i
    + 
    \begin{bmatrix}
    20.361 & 5.876  & 6.740 \\
     0     & 14.697 & 6.795 \\
     0     & 0      & 0.815
    \end{bmatrix} j +
    \begin{bmatrix}
    1.443 & -2.642 &  6.596 \\
    0     & -1.847 & -1.528 \\
    0     &  0     &  5.025
    \end{bmatrix} k.
\end{align*}
}
Thus, the equation can be rewritten as
\begin{equation}
    QRX = B.\label{eq:example1}
\end{equation}

Since $Q$ is a unitary matrix, multiplying both sides of \eqref{eq:example1} by $Q^H$ from the left yields $RX = Q^HB\triangleq \widehat{B}$,
where
{\footnotesize
\begin{align*}
  \widehat{B} =
    & \begin{bmatrix}
     4.981  &  6.383  &  8.076 \\
    -2.019  & -1.922  & -1.440 \\
     11.356 &  10.929 & -9.652
    \end{bmatrix} +
    \begin{bmatrix}
    -1.144 & -7.725  &  6.206 \\
     1.152 &  11.947 & -2.239 \\
    -3.761 &  0.482  & -0.749
    \end{bmatrix} i +
    \begin{bmatrix}
    -6.136 & -7.249 & -9.693 \\
     0.310 & -4.646 &  0.574 \\
     1.391 & -2.620 & -3.994
    \end{bmatrix} j +
    \begin{bmatrix}
     11.385 & -4.261 & -2.859 \\
    -7.851  &  1.655 & -0.404 \\
    -0.653  &  6.826 &  12.841
    \end{bmatrix} k.
\end{align*}
}
Note that $R$ is an upper triangular matrix. Thus, $RX = Q^HB\triangleq \widehat{B}$ can be solved using back substitution:
{\footnotesize
 \begin{align*}
  X =
    & \begin{bmatrix}
    -1.082 & -0.797 &  1.407 \\
    -0.316 & -0.028 &  0.749 \\
     1.846 &  0.845 & -2.243
    \end{bmatrix} +
    \begin{bmatrix}
    -1.116 & -0.337 &  0.999 \\
    -0.423 & -0.961 &  1.907 \\
     0.349 &  1.315 & -0.403
    \end{bmatrix} i +
    \begin{bmatrix}
    -0.456 & -0.123 &  1.793 \\
    -1.222 & -0.449 &  1.810 \\
     0.402 & -1.119 & -1.423
    \end{bmatrix} j +
    \begin{bmatrix}
     0.944 &  0.982 & -1.597 \\
     0.059 & -0.514 &  0.229 \\
    -0.989 & -0.256 &  2.156
    \end{bmatrix} k.
\end{align*}
}
And the relative error is $\frac{\|AX - B\|_F}{\|B\|_F} = 1.2412\times 10^{-15}$.
\fi
%-------------------- 参考文献 --------------------
\begin{thebibliography}{99}

\bibitem[Abłamowicz(2020)]{Abłamowicz2020} R. Abłamowicz, The Moore–Penrose inverse and singular value decomposition of split quaternions, Adv. Appl. Clifford Algebr. 33 (30) (2020)1–20.

\bibitem[Yasemin(2012)]{Yasemin2012} Y. Alag\"oz, K. Oral, and S. Y\"uce. Split quaternion matrices, Miskolc Mathematical Notes 13.2 (2012) 223–232.

\bibitem[Cockle(1849)]{Cockle1849} J. Cockle, On systems of algebra involving more than one imaginary; and on equations of the fifth degree, Phil. Mag. 35 (1849) 434–437.

\bibitem[Gog(2022)]{Gog2022} M. Gogberashvili, (2+1)-Maxwell equations in split quaternions, Physics 4 (1) (2022) 329–363.

\bibitem[Hasebe(2010)]{Hasebe2010} K. Hasebe, Split quaternionic hopf map, quantum hall effect, and twistor theory, Phys. Rev. D 81 (4) (2010) 041702.

\bibitem[TJiang(2015)]{TJiang2015} T. Jiang, Z. Jiang, Z. Zhang, Algebraic techniques for diagonalization of a split quaternion matrix in split quaternion mechanics, J. Math. Phys. 56 (2015) 083509.

\bibitem[Jiang(2018)]{Jiang2018}T. Jiang, Z. Zhang, Z. Jiang, Algebraic techniques for eigenvalues and eigenvectors of a split quaternion matrix in split quaternionic mechanics,Comput, Phys. Comm. 229 (2018) 1–7.

\bibitem[TJiang(2018)]{TJiang2018}T. Jiang, Z. Zhang, Z. Jiang, Algebraic techniques for Schrödinger equations in split quaternionic mechanics, Comput. Math. Appl. 75 (2018)2217–2222.

\bibitem[Le(2022)]{Le2022} E. Legrand, The geometry of dissipative mechanical systems: Using Jacobi manifolds and the split quaternion algebra, Delft University of
Technology, 2022.

\bibitem[Zhuo(2020)]{Zhuo2020} X. Liu and Z. He. On the split quaternion matrix equation $AX= B$, Banach Journal of Mathematical Analysis 14.1 (2020) 228-248.

\bibitem[Yang(2020)]{Yang2020} X. Liu and Y. Zhang. Least squares \(X = {X^{\eta}}^* \) solutions to split quaternion matrix equation \(AX{A^{\eta}}^*= B\), Mathematical Methods in the Applied Sciences 43.5 (2020) 2189-2201.

%\bibitem[Xin(2019)]{Xin2019} X. Liu and Y. Zhang. Consistency of Split Quaternion Matrix Equations $AX^* - XB = CY + D$ and $X - AX^*B = CY + D$. Advances in Applied Clifford Algebras 29 (2019) 1-20.

\bibitem[Z(2022)]{Z2022} Z. Özdemir, A kinematic model of the Rytov’s law in the optical fiber via split quaternions: application to electromagnetic theory, Euro. Phys.J. Plus 137 (6) (2022) 1–13.

\bibitem[mma(2023)]{mma} İ. Öztürk, and M. Özdemir. On geometric interpretations of split quaternions, Mathematical Methods in the Applied Sciences 46.1 (2023) 408-422.

\bibitem[wang(2024)]{wang} K.W. Si, Q.W. Wang, L.M. Xie. A classical system of matrix equations over the split quaternion algebra, Advances in Applied Clifford Algebras 34.5 (2024) 51.

\bibitem[Wang(2023)]{Wang2023} G. Wang, T. Jiang, V.I. Vasil’ev, Z. Guo, An efficient method for Maxwell’s equations with a discrete double-curl operator in split quaternionic electromagnetics, Eur. Phys. J. Plus 341 (138) (2023) 1–6.

\bibitem[Wang(2021)]{Wang2021}G. Wang, T. Jiang, Z. Guo, D. Zhang, A complex structure-preserving algorithm for split quaternion matrix LDU decomposition in split quaternion mechanics, Calcolo 58 (34) (2021) 1–15.

\bibitem[Gang(2024)]{Gang2024}G. Wang, T. Jiang, V. Vasil’ev, and Z. Guo. On singular value decomposition for split quaternion matrices and applications in split quaternionic mechanics, Journal of Computational and Applied Mathematics 436 (2024) 115447.

\bibitem[yuan(2017)]{yuan}S.F. Yuan, Q.W. Wang, Y. Yu. On Hermitian solutions of the split quaternion matrix equation $AXB+CXD=E$, Advances in Applied Clifford Algebras 27 (2017) 3235-3252.

\bibitem[Zhang(2015)]{Zhang2015}Z. Zhang, Z. Jiang, T. Jiang, Algebraic methods for least squares problem in split quaternionic mechanics, Appl. Math. Comput. 269 (2015) 618–625.

\end{thebibliography}

\end{document}